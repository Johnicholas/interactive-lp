

\documentclass{article}

\usepackage{amsmath, amssymb}
\usepackage{verbatim}
\usepackage{url}

% XXX add fp-macros or sth
\newcommand{\lolli}{\multimap}
\newcommand{\tensor}{\otimes}


\title{Towards a proof-theoretic understanding of reactive and interactive
programming}
\author{Chris Martens}
\date{\today}

\begin{document}

\maketitle

\section{Introduction}

This project seeks to establish a proof-theoretic, unifying approach to
encoding and reasoning about reactive and interactive systems.

With cutting-edge programming language technology, we can write complex
specifications of real systems, execute those specifications, and for
certain classes of systems, state and prove theorems about them. These
reasoning tools are based variously on dependent types (LF), substructural
logic (linear, ordered, affine, modal), logic programming,
and focusing.

The systems we are capable of expressing with this technology are, however,
limited to those where we expect either a single execution or a universal
characterization of a class of executions. The specifications represent a
self-contained simulation that runs, computes an answer or set of answers,
and finishes (or continues uninterrupted). A stack machine to evaluate
a programming language is a common example of such a thing: it expects a
closed program, containing all the information it needs to execute, and it
derives (through the rules encoded in logic of the specification language)
an evaluation of that program.

CLF~\cite{CLF} was given that name as the ``concurrent logical framework'', and
its proof theory is capable of expressing a wide range of concurrent
systems, so long as we think of {\em concurrent} as a synonym for {\em
nondeterministic}. What it fails to do is let us encapsulate a set of rules
as a distinct process, reason about that process with respect to the
interfaces of others, prove properties about its own interface, or
otherwise give meaning to the partial program.

I'm interested in using the proof-theoretic techniques present in CLF as a
starting point, extending and modifying them to reason about {\em reactive}
and {\em interactive} programs, i.e. {\em partial} specifications, that may
be written and reasoned about separately and compositionally.


\section{Problem domain: reactive and interactive systems}

A {\em reactive system} is one that accepts external inputs and behaves
differently upon receipt of those inputs; an {\em interactive system} may
respond more rapidly or proactively. Examples are everywhere. In computing,
they manifest as the operating system, shell, a programming language's
REPL, web browser, a game, or any application that accepts user input.
Examples (ostensibly) outside computing include biological systems (living
organisms), thermostats, other home appliances, and nuclear plants.  In
fact, I might guess that the {\em vast majority} of real systems in the
worlds that we want to reason about are reactive or interactive.

The ``hello world'' example of a reactive system is a toggle or
lightswitch. An external signal can flip it from on to off and vice versa.
Already, we can dissect such a problem into three parts: its input channel (the
switch), its internal state (whether it is on or off), and its output
channel (the bulb, or a printout of the state). This ontology is related to
the common ``model, view, controller'' (MVC) way of conceptualizing
interactive programs.

While the world has plenty of tools for writing such systems, very few have
been developed for describing them in such a way that we can formally
reason about their properties (for example, Hennessy-Milner sorts of
properties, like that as long as the switch keeps receiving signals, there
will always be a future in which the light is off).

As soon as several input or output channels are allowed, complexity
skyrockets. We must worry about race conditions, shared state, and every
other worry ever to plague concurrency. In these settings, the need for
reasoning tools is even more dire.

\section{Starting point: linear logic programming}

% XXX explain Focussing; substructuralness; fwd/backward chaining

Here I show how far one can get with encoding several examples in CLF
and explain their limitations that I wish to resolve.

\subsection{Separability}

Here is an encoding of the toggle example in CLF:

\verbatiminput{../examples/toggle-1.clf}

By convention, names preceded by \verb|sig| can be interpreted as
``signals'', either inputs or outputs. \verb|sigtoggle| represents some
signal that comes from the outside. Note that no rule in this program
generates a \verb|sigtoggle|, they only consume it.

A CLF \verb|#trace| of this program is capable of seeding it with any
initial state for the switch and a finite number $n$ of toggle signals. The
execution of such a trace will carry out those $n$ toggle flips and
terminate. This is fine for writing tests of individual simulations, but
doesn't simulate the {\em dynamic} adding of signals to the context.

Instead, I'd like to write a separate process

\verbatiminput{../examples/toggle-2.clf}

That takes an initial entry point and sends a toggle, waits for the toggle
to fire, and sends another toggle, repeatedly.

Then seeding the trace with

\verbatiminput{../examples/toggle-3.clf}

creates an infinite (but coinductively characterizeable) whole
program that runs infinitely.

Indeed, I can put all of these definitions in the same file and execute the
whole program in CLF. But what I can't do is encapsulate the pieces in any
meaningful way. So far in the code, I can identify for each rule which
``process'' it belongs to, on the basis of which signals are on the left of
a lolli and which are on the right. But I could of course add a rule like

\begin{verbatim}
sigtoggled -o {sigtoggled}
\end{verbatim}

or

\begin{verbatim}
switch on -o {switch off}
\end{verbatim}

and it wouldn't be clear to which process they belong. Further, supposing
we extend the program such that the toggle (which would no longer simply be
a toggle) does a lot more internal computation, we would like the behavior
of the program to avoid considering rules that manipulate that logic while
we are ``inside a different process'', i.e. when a \verb|sigtoggled| is
generated, we should know that we need not (at this time) consider any
rules in the toggle's program.

Thus I claim this leads to the first desired property of a framework for
interactivity: {\em separability}.

\subsection{Contingent transitions}

%% Add the document
In a separate document~\footnote{http://www.cs.cmu.edu/~cmartens/roar.pdf}
(also found in the github directory with this one), I demonstrate
that linear logic does not adequately capture a notion of {\em contingent
transition} or {\em read-only access to resources}. The idea is that you'd
like to only fire a rule $A \lolli B$ under the condition $C$, where $C$ is
a set of linear resources that you demand of your current state (though of
course the initial confusion to resolve is what we mean by ``current
state''). You might initially think to encode such a rule as $A\tensor C \lolli B
\tensor C$, but this doesn't actually preserve the predondition as we might
expect: $A \tensor (C_1\oplus C_2) \lolli B \tensor (C_1 \oplus C_2)$ in
the context with $C_1, A$ can eventually transition to $(C_1\oplus C_2),
B$, but the information about which disjunct we started with is lost.

\subsection{Broadcast}

In the past several weeks I have begun to investigate a wider range of
communication patterns that users of existing concurrent languages expect
to express. One such pattern is that a process may receive or send the same
signal from or to multiple sources.

In a blog post
\url{http://lambdamaphone.blogspot.com/2012/11/interactivity-week-3.html} I
present a notion of broadcast that CLF is incapable of expressing. The idea
is that we can of course always produce a finite, tensored-together group
of signals, but we can't describe that set generally -- adding a new
``neighbor'' to the signal graph amounts to editing every rule that sends
the signal.

We can write
\begin{verbatim}
handle : sigout N -o {Pi x. nbr x N -o sigin x}.
\end{verbatim}

...but the \verb|Pi| in that rule behaves more like an infinitary 
$\&$, allowing us to instantiate it at just one node, rather than what we
really want, an infinitary $\tensor$.

This in some sense just a special case of the Separability
criterion, but one that particularly suggests some interesting proof
theory.

\section{Target examples}

Here I will outline some target examples of interactive programs I would
like to be able to account for. I would like to focus on one of these
examples specifically for my thesis project, but I have not decided which
one.

\subsection{Human-computer interaction}

Examples falling under this umbrella include REPLs, games, and GUIs. An
example I have been thinking about a lot specifically is interactive
fiction or ``text adventure'' games, since they eliminate the overhead of
graphical rendering. A talk I gave at
OBT'12~\footnote{http://www.cs.cmu.edu/~cmartens/obt-talk.pdf} outlines my thoughts
toward using logic programming for this domain. To summarize, the
high-level architecture of a game (text or otherwise) is an event loop with
three steps:

\begin{enumerate}
\item Render the current state of the game.
\item Accept input from the player.
\item Process the input and use internal logic to modify the game state.
\end{enumerate}

This domain offers a trove of interesting questions about how to represent
the game world, which is assembled in terms of spatial relationships --
usually a network of rooms that the player moves between -- each having
several independent interactive components, some of which can be carried
and moved from place to place in addition to changing ``external'' state.
Many player actions are contingent on certain properties of the state for
success, which invokes the read-only access problem.

Aside from those questions, the key thing missing from Celf is the ability
to interrupt the automatic search procedure during forward-chaining to
allow a user-selected rule to fire (from a specially designated subset).

\subsection{Signal processing for audio and graphical visualizations}
(TODO: fill these in)

\subsection{Connecting web APIs}

\subsection{Scripting processes on UNIX}

\section{Related work}

Puredata (pd) (cite) is a language for declaratively describing signal
interactions as a graph: nodes of various data types and edges between
their output and input ports. It is commonly used for programming
generative sound, music, and visualizations. Being able to encode, execute
and reason about the idioms present in such a language has been and I
believe will continue to be fruitful in presenting challenges and examples.

Functional Reactive Programming (cite) is probably the most developed attempt at 
declarative programming for interactive settings. It is implemented mainly
as libraries in Haskell and as Elm (cite), a new FRP language that's
attractive because of the ability to write and run it in a browser. The
fundamental notion in FRP is {\em time-varying values}; functions over such
values behave like stream transformers.

A group at CMU~\cite{RelatingReasoning} recently explored the connection between the proof
theory of linear logic (context transitions) and the reasoning
methodologies used in process calculi a la Milner. An intermediate
semantics they give uses a relation

\[
\Delta \longrightarrow^{\sigma} \Delta'
\]

Where $\sigma$ is an ``external'' signal; there is a special symbol to
represent internal ``silent'' transitions. This system is a proof device,
but as a stepping stone between linear logic and process calculi it seems
compelling, particularly as I want to frame certain signals as external.

Separately, also at CMU, Umut Acar and his student Stefan Muller submitted
an NSF GRFP proposal to explore interactivity in the context of
self-adjusting computation (cite). Their idea is that since SAC is a
framework for re-running pieces of computation on new partial inputs, a
programmatically-repeated SAC could efficiently process
interactively-changing inputs. I plan to pay close attention to their work
along these lines to understand how it lines up with the proof-theoretical
viewpoint.

Andre Platzer's work on dynamic and hybrid logics (e.g. cite) also seems
relevant at least in the sense that dynamic systems receive external
inputs, and dynamic logic seems related to Henessy-Milner, although the
hybrid setting is further challenged by needing to deal with continuous
signals and thus differential equations. FRP has been studied as an account
of programming hybrid systems.~\cite{Hudak2003}

\section{Proposed work}

First, I intend to do further reading and investigation of FRP, SAC,
puredata and other communication graph languages, while simultaneously
encoding examples in CLF and imagined proto-syntax for a distinct language.
I also intend to read about Bunched Implications (BI) and their proposal
for linear operators that might be relevant to this setting.

Then I plan to investigate what kind of metatheory or program-specific
properties we desire about these programs, starting with standard process
calculus metatheoretic techninques, including bisimulation, coinduction,
and Henessy-Milner logic. I will sketch how I plan to approach this sort of
reasoning within my framework.

I propose to build a prototype implementation of this framework and encode
many examples.

\section{Goals and evaluation criteria}

To summarize the criteria I have in mind for an interactivity framework, it
should be capable of expressing
\begin{itemize}
\item Separability/composability: distinct processes can be encoded, run,
and reasoned about in isolation, even if they need external signals to make
progress
\item Contingent transitions, i.e. combine deductive, reflective reasoning
on current states with transitions between those states
\item Broadcast, and other multi-channel communication patterns.
\end{itemize}


\bibliographystyle{plain}
\bibliography{preproposal}

\end{document}
